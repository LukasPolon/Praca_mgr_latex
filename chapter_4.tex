%% ------- ROZDZIAŁ 4 ------- %%

\chapter{Testy aplikacji}

\section{Cel przeprowadzonych analiz}

Wykorzystując napisaną na potrzeby niniejszej pracy aplikację przeprowadzono testy, które zmierzają do porównania wybranych modeli regresji dostępnych w pakiecte \textit{Scikit-learn}.\\

Testy przeprowadzone zostały na dwóch zbiorach danych: cen giełdowych firmy Microsoft w zakresie od 2017-01-01 do 2017-10-30, oraz cen giełdowych firmy Intel w zakresie od 2017-01-01 do 2017-04-30.
W dlaszej części zbiory te będą nazywane odpowiednio: zbiór szeroki i zbiór wąski.
Dane wykorzystane do testów były cenami otwarcia.\\

Przeprowadzono osobne testy dla każdej z trzech wartości procentowych ilości danych uczących w stosunku do ilości danych testowych: 20\%, 50\% oraz 80\%.\\

Wybrane modele regresji to:
\begin{itemize}
 \item Regresja Liniowa
 \item Regresja Grzbietowa (KRR)
 \item Regresja Wektorów Nośnych (SVR)
 \item Regresja Procesu Gaussa (GPR)
\end{itemize}

Reasumując, dla każdej z metod regresji wykonano sześć testów.\\

Celem testów jest porównanie modeli regresji dostępnych w pakiecie \textit{Scikit-learn} i wyciągnięcie wniosków dotyczących:
\begin{itemize}
 \item dokładności predyckji modeli w zależności od ilości danych uczących oraz całkowitej ilości danych
 \item zdolności modeli do reprezentacji trendu
 \item wpływu zmiany ilości danych uczących na dopasowanie modeli
 \item wpływu całkowitej ilości danych na dopasowanie modeli
\end{itemize}

\section{Testy zbioru danych: Microsoft}

\subsection{Informacje ogólne}
\subsection{Regresja liniowa}
\subsection{Regresja Grzbietowa}
\subsection{Regresja Wektoróœ Nośnych}
\subsection{Regresja Procesu Gaussa}
\subsection{Podsumowanie}

\section{Testy zbioru danych: Intel}

\subsection{Informacje ogólne}
\subsection{Regresja liniowa}
\subsection{Regresja Grzbietowa}
\subsection{Regresja Wektoróœ Nośnych}
\subsection{Regresja Procesu Gaussa}
\subsection{Podsumowanie}

\section{Wnioski}
