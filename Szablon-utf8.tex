%%====================================================================%%
%%                AKADEMIA GÓRNICZO-HUTNICZA W KRAKOWIE
%%                    WYDZIAŁ MATEMATYKI STOSOWANEJ
%%                          PRACA MAGISTERSKA
%%
%%       Autor : ----
%% Specjalność : ----
%%   Nr albumu : ----
%%    Promotor : ----
%%  Data (wer) : DD.MM.YYYY ...
%%
%%--------------------------------------------------------------------%%
%%
%%
%% ---------------
%%   PLIK GŁÓWNY
%% ---------------
%%
\documentclass[man,mfi|min|mpt|mok|mub|mza|miz|bsp]{mgrwms}

%%
%  Szablon dla kodowania UTF-8 przygotowany z pierwotnego szablonu
%  przy pomocy programu Gżegżółka 5.51.03 (www.gzegzolka.com)
%
%  Obowiązkowe opcje klasy:
%  - woman|man : wersja odpowiednio dla Pań i Panów,
%  - mfi|min|mpt|mok|mub|mza|miz|bsp : specjalności według skrótów:
%      * mfi :  Matematyka finansowa
%      * min :  Matematyka w informatyce
%      * mpt :  Matematyka w naukach technicznych i przyrodniczych
%      * mok :  Matematyka obliczeniowa i komputerowa
%      * mub :  Matematyka ubezpieczeniowa
%      * mza :  Matematyka w zarządzaniu
%      * miz :  Matematyka w informatyce i zarządzaniu
%      * bsp :  Bez specjalności
%  (opis pozostałych opcji znajduje się w dokumentacji 'UserGuide.pdf')
%%
%% ------- PAKIETY ------- %%
%%
\usepackage[utf8]{inputenc}      % kodowanie UTF-8
% \usepackage{amsmath}           % łatwiejszy skład matematyki
% \usepackage{amssymb}
% \usepackage{setspace}          % zmiana interlinii (np. 1,5 wiersza odstępu)
% \usepackage{makeidx}           % odkomentować, jeśli dołączony jest indeks
%
%  <... pozostale pakiety wedle uznania ...>
%
\usepackage{polski}         % pakiet polonizacyjny

\newcommand{\source}[1]{{Źródło: {#1}} }

\usepackage{color}
\usepackage{graphicx}
\usepackage{hyperref}
\usepackage{listings}

\usepackage{amsmath}
\usepackage{nccmath}

\usepackage{dirtree}
\usepackage{float}

\lstset{
language=Python,
basicstyle=\small\sffamily,
numbers=left,
numberstyle=\tiny,
frame=tb,
%columns=fullflexible,
columns=fixed,
showstringspaces=false
}

%\titleclass{\subsubsubsection}{straight}[\subsection]

%\newcounter{subsubsubsection}[subsubsection]
%\renewcommand\thesubsubsubsection{\thesubsubsection.\arabic{subsubsubsection}}
%\renewcommand\theparagraph{\thesubsubsubsection.\arabic{paragraph}} % optional; useful if paragraphs are to be numbered

%\titleformat{\subsubsubsection}
%  {\normalfont\normalsize\bfseries}{\thesubsubsubsection}{1em}{}
%\titlespacing*{\subsubsubsection}
%{0pt}{3.25ex plus 1ex minus .2ex}{1.5ex plus .2ex}

\makeatletter
\renewcommand\paragraph{\@startsection{paragraph}{5}{\z@}%
  {3.25ex \@plus1ex \@minus.2ex}%
  {-1em}%
  {\normalfont\normalsize\bfseries}}
\renewcommand\subparagraph{\@startsection{subparagraph}{6}{\parindent}%
  {3.25ex \@plus1ex \@minus .2ex}%
  {-1em}%
  {\normalfont\normalsize\bfseries}}
\def\toclevel@subsubsubsection{4}
\def\toclevel@paragraph{5}
\def\toclevel@paragraph{6}
\def\l@subsubsubsection{\@dottedtocline{4}{7em}{4em}}
\def\l@paragraph{\@dottedtocline{5}{10em}{5em}}
\def\l@subparagraph{\@dottedtocline{6}{14em}{6em}}
\makeatother

\setcounter{secnumdepth}{4}
\setcounter{tocdepth}{4}


\graphicspath{ {/home/lukas/mgr/MData/mData_latex/schema/szablon/images/} }
% \usepackage[T1]{fontenc}  % lub z opcją 'QX', wymagane dla pakietu 'lmodern'
% \usepackage{lmodern}      % zalecane czcionki LModern
%%
%  Dodatkowe polecenia które trzeba umieścić w preambule:
%
% \makeindex  % jeśli dołączony jest indeks (nieobowiązkowy)
% \includeonly{...,...,...,...}  % dla warunkowej kompilacji
%%
%% BiBTeX (opcjonalnie)
% \bibliographystyle{ddabbrv}  % lub np. plabbrv, plalpha, plplain ...
% \nocite{*}
%%
%%
\begin{document}
%%
%%
%% ---------------
%%   NASZE MAKRA
%% ---------------
%%
%  Miejsce na nasze makra. Innym sposobem może być
%  umieszczenie ich w osobnym pliku i wczytanie poleceniem \input{----}
%%
%%
%% ---------------------------------
%%   METRYCZKA PRACY i SPIS TREŚCI
%% ---------------------------------
%%
\title{Analiza danych giełdowych przy pomocy narzędzi dostępnych w pakiecie scikit-learn}
\author{Łukasz Połoń}
\promotor{dr inż. Piotr Błaszyński}
\nralbumu{24942}
\slowakluczowe{Python, Scikit-learn, Giełda Papierów Wartościowych, Kivy, Matplotlib, Pandas, Analiza regresji}
\keywords{Python, Scikit-learn, Stock Market, Kivy, Matplotlib, Pandas, Regression analysis}
\maketitle
%%
%%
\tableofcontents
%%
%%
%% ----------------------
%%   STRESZCZENIA PRACY
%% ----------------------
%%
%% ------- POLSKIE ------- %%
%%
\begin{streszczenie}
%%
Analiza danych giełdowych udostępnianych przez Giełdy Papierów Wartościowych od samego początku stanowi wyzwanie inwestorów.
Jej wynik pozwala podejmować decyzje finansowe, które obarczone są mniejszym ryzykiem straty, a także pozwala przewidywać przyszłe trendy i zmiany kursów.
Podział na anlizę fundamentalną orez techniczną jest istotny z punktu widzenia narzędzi stosowanych do jej przeprowadzenia.
Analiza techniczna, której elementy są częścią niniejszej pracy, skupia się głównie na analizie wykresów surowych danych giełdowych, ich opisie oraz określaniu trendów.
Jeden z rodzajów danych możliwych do analizy to ceny akcji poszczególnych spółek.
Dane te, udostępnione przez API \textit{Yahoo Finance}, są przedmiotem przeprowadzanych analiz zamieszczonych w niniejszej pracy.\\

Język programowania Python jest interpretowanym, dynamicznie typowanym językiem wysokiego poziomu.
Jego cechy, wśród których można wyróżnić przede wszystkim szybkie wytwarzanie oraz utrzymywanie kodu oraz mnogość dodatkowych bibliotek, 
stwarzają bardzo dobre warunki do wykorzystania go do przeprowadzania analiz statystycznych, matematycznych oraz finansowych.\\

Biblioteka \textit{Scikit-learn} jest jednym z najbardziej rozbudowanych pakietów języka programowania Python, służących do analizy danych metodami uczenia maszynowego.
Udostępnia ona klasy, dzięki którym można wykonać analizy regresji liniowych, nieliniowych, czy również dokonać klasyfikacji.\\

Dla celów niniejszej pracy sporządzono aplikację, która pobiera dane giełdowe określonej spółki, a następnie dokonuje analizy regresji podanym przez użytkownika modelem spośród: regresji liniowej, grzbietowej, wektorów nośnych oraz procesu Gaussa.
Celem pracy jest porównanie tych modeli regresji oraz wyciągnięcie wniosków dotyczących ich możliwości predykcyjnych, a także anlizy trendów.

%
%  <Treść streszczenia po polsku>
%
%%
\end{streszczenie}
%%
%%
\begin{abstract}
%%
%
Stock data analysis, which are shared by Stock Exchanges, from the very beginning is a challenge for many investors.
Its results allow to make financial decisions, which are less burdned with risk of loss. Besides, it allows to predict future trends and course changes.
Division into technical analysis and fundamental analysis is very important from the point of view of tools, which are used to carry out the analysis.
Technical analysis, which elements are part of this work, focuses mainly on plot data anlysis, their description and trends determination.
One of data types, which are possible to analyze are stock market's shares pices.
That data, which is shared by \textit{Yahoo Finance} API, is the subject of analysis included in this work.\\

Python is a high-level, interpreted and dynamically-typed programming language.
Its features, among which we can distinguish fast code production and maintenance, as well as high amount of additional libraries, create very good conditions for using it for statistical, mathematical and financial analysis.\\

\textit{Scikit-learn} library is one of the most extensive Python packages used for machine learning analysis.
It provides classes, by which is a possibility to do the linear and non-linear regression analysis, as well as classification analysis.\\

Application has been created for this work purposes. This application collects stock data and analyzes it by one of four regression methods: linear, ridge, supprot vector and Gauss process.
The purpose of this work is comparison of these regression models and to draw conclusions about its prediction capabilites, as well as trends analysis.


%%
\end{abstract}
%%
%% ------- PODZIĘKOWANIA ------- %%
%%
%  <Jeśli zachodzi taka potrzeba, można dodać tutaj tekst podziękowania,
%   dedykacji itp.>
% \newpage
% \thispagestyle{empty}
%  <Treść podziękowania z odpowiednim formatowaniem>
%%
%%
%% ----------------------
%%   GŁÓWNY TEKST PRACY
%% ----------------------
%%
%% ------- WSTĘP ------- %%
%%
\begin{wstep}[Wprowadzenie]  % ew. np. \begin{wstep}[Wprowadzenie]
%%
Giełda Papierów Wartościowych to instytucja, która prowadzi działalność w zakresie organizacji obrotu papierami wartościowymi i instrumentami finansowymi\cite{gpw}.
W praktyce spełnia ona rolę pośrednika finansowego pomiędzy kupującym, a sprzedającym papiery wartościowe.
Dane generowane przez giełdę poddawane są ciągłym analizom, w szczególności w celu dostarczenia informacji potrzebnych do właściwego zarządzania kapitałem.\\

Istnieją dwie główne metody analizy danych giełdowych: analiza fundamentalna i analiza techniczna\cite{basics_of_tech_analysis}.
Pierwsza z nich polega na analizie faktycznej kondycji finansowej podmiotu, podczas gdy druga ma za zadanie prognozowanie przyszłych wartości wskaźników lub cen na podstawie zebranych danych.\\

Temat tej pracy podejmuje opisanie i przetestowanie wybranych metod analitycznych dostępnych w pakiecie \textit{Scikit-learn}.
Pakiet ten jest biblioteką języka programowania Python umożliwiającą wysokopoziomowe przetwarzanie danych.
Udostępnia wiele algorytmów klasyfikacji, regresji oraz uczenia maszynowego, które mogą zostać wykorzystane do przeprowadzania obliczeń między innymi na potrzeby analizy technicznej, której to elementy zostaną tutaj przedstawione.

%%
\end{wstep}
%%
\input{chapter_1}

%% ------- ROZDZIAŁ 2 ------- %%

\chapter{Zastosowanie języka programowania Python w analizie rynków finansowych}
\section{Cechy charakterystyczne języka Python}
\section{Python w obliczeniach analitycznych}
\section{Pakiet Scikit-learn}
\subsection{Cel i przeznaczenie pakietu}
\subsection{Modele liniowe}
\subsection{Wybrane modele klasyfikacji i regresji}


%% ------- ROZDZIAŁ 3 ------- %%

\chapter{Przedstawienie aplikacji}

\section{Podstawowe założenia}

\section{Zastosowane pakiety języka Python}

\subsection{Kivy}

\subsection{Pandas i Matplotlib}

\subsection{Scikit-learn}

\section{Opis funkcjonalności aplikacji}

\subsection{Okno opcji}

\subsection{Okno analizy regresji}

\section{diagramy UML}

%% ------- ROZDZIAŁ 4 ------- %%

\chapter{Testy aplikacji}

\section{Cel przeprowadzonych analiz}

Wykorzystując napisaną na potrzeby niniejszej pracy aplikację przeprowadzono testy, które zmierzają do porównania wybranych modeli regresji dostępnych w pakiecte \textit{Scikit-learn}.\\

Testy przeprowadzone zostały na dwóch zbiorach danych: cen giełdowych firmy Microsoft w zakresie od 2017-01-01 do 2017-10-30, oraz cen giełdowych firmy Intel w zakresie od 2017-01-01 do 2017-04-30.
W dlaszej części zbiory te będą nazywane odpowiednio: zbiór szeroki i zbiór wąski.
Dane wykorzystane do testów były cenami otwarcia.\\

Przeprowadzono osobne testy dla każdej z trzech wartości procentowych ilości danych uczących w stosunku do ilości danych testowych: 20\%, 50\% oraz 80\%.\\

Wybrane modele regresji to:
\begin{itemize}
 \item Regresja Liniowa
 \item Regresja Grzbietowa (KRR)
 \item Regresja Wektorów Nośnych (SVR)
 \item Regresja Procesu Gaussa (GPR)
\end{itemize}

Reasumując, dla każdej z metod regresji wykonano sześć testów.\\

Celem testów jest porównanie modeli regresji dostępnych w pakiecie \textit{Scikit-learn} i wyciągnięcie wniosków dotyczących:
\begin{itemize}
 \item dokładności predyckji modeli w zależności od ilości danych uczących oraz całkowitej ilości danych
 \item zdolności modeli do reprezentacji trendu
 \item wpływu zmiany ilości danych uczących na dopasowanie modeli
 \item wpływu całkowitej ilości danych na dopasowanie modeli
\end{itemize}

\section{Testy zbioru danych: Microsoft}

\subsection{Informacje ogólne}
\subsection{Regresja liniowa}
\subsection{Regresja Grzbietowa}
\subsection{Regresja Wektoróœ Nośnych}
\subsection{Regresja Procesu Gaussa}
\subsection{Podsumowanie}

\section{Testy zbioru danych: Intel}

\subsection{Informacje ogólne}
\subsection{Regresja liniowa}
\subsection{Regresja Grzbietowa}
\subsection{Regresja Wektoróœ Nośnych}
\subsection{Regresja Procesu Gaussa}
\subsection{Podsumowanie}

\section{Wnioski}



%%% ------- ROZDZIAŁ 4 ------- %%

\chapter{Testy aplikacji}

\section{Cel przeprowadzonych analiz}

Wykorzystując napisaną na potrzeby niniejszej pracy aplikację przeprowadzono testy, które zmierzają do porównania wybranych modeli regresji dostępnych w pakiecte \textit{Scikit-learn}.\\

Testy przeprowadzone zostały na dwóch zbiorach danych: cen giełdowych firmy Microsoft w zakresie od 2017-01-01 do 2017-10-30, oraz cen giełdowych firmy Intel w zakresie od 2017-01-01 do 2017-04-30.
W dlaszej części zbiory te będą nazywane odpowiednio: zbiór szeroki i zbiór wąski.
Dane wykorzystane do testów były cenami otwarcia.\\

Przeprowadzono osobne testy dla każdej z trzech wartości procentowych ilości danych uczących w stosunku do ilości danych testowych: 20\%, 50\% oraz 80\%.\\

Wybrane modele regresji to:
\begin{itemize}
 \item Regresja Liniowa
 \item Regresja Grzbietowa (KRR)
 \item Regresja Wektorów Nośnych (SVR)
 \item Regresja Procesu Gaussa (GPR)
\end{itemize}

Reasumując, dla każdej z metod regresji wykonano sześć testów.\\

Celem testów jest porównanie modeli regresji dostępnych w pakiecie \textit{Scikit-learn} i wyciągnięcie wniosków dotyczących:
\begin{itemize}
 \item dokładności predyckji modeli w zależności od ilości danych uczących oraz całkowitej ilości danych
 \item zdolności modeli do reprezentacji trendu
 \item wpływu zmiany ilości danych uczących na dopasowanie modeli
 \item wpływu całkowitej ilości danych na dopasowanie modeli
\end{itemize}

\section{Testy zbioru danych: Microsoft}

\subsection{Informacje ogólne}
\subsection{Regresja liniowa}
\subsection{Regresja Grzbietowa}
\subsection{Regresja Wektoróœ Nośnych}
\subsection{Regresja Procesu Gaussa}
\subsection{Podsumowanie}

\section{Testy zbioru danych: Intel}

\subsection{Informacje ogólne}
\subsection{Regresja liniowa}
\subsection{Regresja Grzbietowa}
\subsection{Regresja Wektoróœ Nośnych}
\subsection{Regresja Procesu Gaussa}
\subsection{Podsumowanie}

\section{Wnioski}


%%% ------- ROZDZIAŁ 5 ------- %%

\chapter{Wnioski}



%% ------- ZAKOŃCZENIE ------- %%
%%
% \begin{zakonczenie}  % ew. np. \begin{zakonczenie}[Podsumowanie]
%%
%
%  <Treść zakończenia, podsumowania (jeśli jest taka potrzeba umieszczenia)>
%
%%
% \end{zakonczenie}
%%
%%
%% -----------------------
%%   ROZDZIAŁY DODATKOWE
%% -----------------------
%%
% \appendix
%%
%% ------- DODATEK A ------- %%
%%
% \chapter{----}
%%
%
%  <Treść dodatku A, może być umieszczona w innym pliku
%   i ładowana przez \input{----} lub \includeonly{...,...}>
%
%%
%%
%% ----------------
%%   BIBLIOGRAFIA
%% ----------------
%%
%  (Możemy użyć środowiska 'thebibliography' lub też bazy BiBTeX)
%%
%% BiBTeX (opcjonalnie)
% \bibliography{<pliki bib>}
%%
\begin{thebibliography}{88}  % ew. np. '8' jeśli liczba pozycji < 10
%% Przykładowe wrzuty

\bibitem{gpw} Tobiasz Maliński, 
\textit{Giełda Papierów Wartościowych Dla Bystrzaków}, Helion 2016.

\bibitem{basics_of_tech_analysis} Justin Kuepper,
\textit{Basics of Technical Analysis}, 19 Kwiecień 2017.
https://www.investopedia.com/university/technical/

\bibitem{stock_market} Investopedia,
\textit{Stock Market}, 20 Listopad 2017.\\
https://www.investopedia.com/terms/s/stockmarket.asp

\bibitem{ustawa_22_03_1991_papiery_wartosciowe} Prawo o publicznym obrocie papierami wartościowymi i funduszach powierniczych
\textit{Ustawa z dnia 22 marca 1991r.}, Art 2.

\bibitem{akcje} Roman Ciepiela, Piotr Pytlik, Magda Wiernusz,
\textit{Encyklopedia Zarządzania}\\
25 Październik 2016.\\
mfiles.pl/pl/index.php/Akcje

\bibitem{obligacje} Roman Ciepiela, Szymon Kułakowski, Sabina Blok,
\textit{Encyklopedia Zarządzania}, 13 Lipiec 2017
mfiles.pl/pl/index.php/Obligacje

\bibitem{ekonometria} Małgorzata Łuniewska
\textit{Ekonometria Finansowa: Analiza rynku kapitałowego}, Warszawa 2008
Wydawnictwo Naukowe PWN

\bibitem{tech_analysis_pillars} Stockopedia
\textit{Technical Analysis (Part 1): History, Theory and Philosophy}\\
Kwiecień 2017\\
www.stockopedia.com/content/technical-analysis-part-1-history-theory-and-philosophy-179598/

\bibitem{ekonometria_model} G.S. Maddala
\textit{Ekonometria} Warszawa 2006\\
Wydawnictwo Naukowe PWN

\bibitem{analiza_fundamentalna_techniczna} Mariusz Czekała
\textit{Analiza fundamentalna i techniczna} Wrocław 1997\\
Wydawnictwo Akademii Ekonomicznej im. Oskara Langego we Wrocławiu

\bibitem{systemy_uczace_sie} Jacek Koronacki, Jan Ćwik
\textit{Statystyczne Systemy Uczące Się} Warszawa 2005\\
Wydawnictwa Naukowo-Techniczne

\bibitem{decision_tree_classification} Emil Lundkvist
\textit{Decision Tree Classification and Forecasting of Pricing Time Series Data} Stockholm 2014\\
Master’s Degree Project

\bibitem{intro_machine_learning} Alex Smola
\textit{Introduction to Machine Learning} 2008\\
Cambridge University Press

\bibitem{bayes} Paweł Cichosz
\textit{Systemy Uczące Się} Warszawa 2000\\
Wydawnictwo Naukowo-Techniczne Warszawa

\bibitem{drzewa_decyzyjne} Mirosław Krzyśko
\textit{Systemy uczące się} Warszawa 2008
Wydawnicatwa Naukowo-Techniczne

\bibitem{statystyka} Stanisław M. Kot
\textit{Statystyka} Warszawa 2011
Difin SA

\bibitem{analiza_danych} Siegmund Brandt
\textit{Analiza Danych} Warszawa 1998
Wydawnictwo Naukowe PWN

\bibitem{learning_python} Fabrizio Romano
\textit{Learning Python} 2015
Packt Publishing

\bibitem{numpy_ug} NumPy Community
\textit{NumPy User Guide} Release 1.13.0

\bibitem{scipy_doc} SciPy.org
\textit{SciPy Documentation}
https://docs.scipy.org/doc/scipy/reference/

\bibitem{scikit_article} Fabian Pedregosa
\textit{Scikit-learn: Machine Learning in Python}
Journal of Machine Learning Research 12 (2011

\bibitem{scikit_doc} scikit-learn.org
\textit{Sciki-learn Documentation}
http://scikit-learn.org/stable/documentation.html

\bibitem{mastering_stat} Gavin Hackeling
\textit{Mastering Machine Learning with scikit-learn}
Packt Publishing


%%
\end{thebibliography}
%%
%%
%% ----------------------------
%%   DODATKOWE ELEMENTY PRACY
%% ----------------------------
%%
%  Tutaj umieścimy dodatkowe (niebowiązkowe) cześci pracy:
%  - Spis skrótów, symboli i oznaczeń (jako \chapter*{...})
%  - Spis ilustracji (\listoffigures)
%  - Spis tablic (\listoftables)
%  - Skorowidz (nieobowiązkowy, umieszczenie ułatwia lekturę pracy)
%
% \chapter*{Spis symboli i oznaczeń}  % ew. podobny tytuł
%  
%   <Tabela ze spisem symboli, oznaczeń itp.>
%
\listoffigures
%
% \listoftables

%\renewcommand{\lstlistlistingname}{Spis listingów}
%\lstlistoflistings
%\addcontentsline{toc}{chapter}{Spis listingów}

%
%\printindex  % jeżeli do pracy dołączony jest indeks
%%
%%
\end{document}
%%
%% [Wersja dla: 2011/09/06 v1.06]
%%====================================================================%%
