%% ------- ROZDZIAŁ 2 ------- %%

\chapter{Zastosowanie języka programowania Python w analizie rynków finansowych}
Python jest językiem programowania wysokiego poziomu, charakteryzujący się przede wszystkim wysoką klarownością i czytelnością kodu.
Jest to język interpretowany, co w odróżnieniu od języków kompilowanych pozwala na bardzo szybkie tworzenie i testowanie kodu.
Wadą tego rozwiązania jest niestety spadek wydajności oraz zwiększone zużycie pamięci i procesora, jednak zastosowania praktyczne Pythona zazwyczaj pozwalają na poniesienie tego typu kosztów.

Python został stworzony w 1989 roku przez Guido van Rossum, a do dzisiaj rozwijany jest jako projekt Open Source i zarządzany przez organizację non-profit Python Software Foundation.
Jego specyficzna struktura oraz cechy takie jak dynamiczne typowanie, automatyczna zarządzanie pamięcią, przenośność, czy duża czytelność i prostota kodu, 
umożliwiają bardzo szybkie wytwarzanie i utrzymywanie aplikacji.\\

Biblioteka standardowa języka Python zawiera wiele użytecznych modułów i gotowych rozwiązań, które wspomagają szybką i efektywną implementację kodu.
Ponadto dostępny jest \textit{Python Package Index} (PyPI) - zbiór paczek zewnętrznych, tworzonych przez niezależnych programistów, dystrybuowanych na licencjach Open Source.
Dzięki takiej mnogości pakietów i modułów język Python może być wykorzystywany w wielu projektach, łącząc różne technologie i dziedziny informatyki.
Jednym z przykładów wykorzystania tego języka jest tworzenie aplikacji internetowych za pomocą frameworku Django.
Łączy on ze sobą różne technologie wykorzystywane przy tworzeniu serwisów internetowych, zapewniając bardzo dobry mechanizm back-endowy oraz wygodne środowisko.\\

Ze względu na 


\section{Cechy charakterystyczne języka Python}
\section{Python w obliczeniach analitycznych}
\section{Pakiet Scikit-learn}
\subsection{Cel i przeznaczenie pakietu}
\subsection{Modele liniowe}
\subsection{Wybrane modele klasyfikacji i regresji}
